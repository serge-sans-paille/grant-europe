\begin{sitedescription}{UB}
% PIC: 
% see: http://ec.europa.eu/research/participants/portal/desktop/en/orga

% See ../proposal.tex, section Members of the Consortium for a
% complete description of what should go there

\begin{itemize}
\item INRIA, LaBRI, IMB
\item journal de th\'eorie des nombres
\item Plafrim and a mesocentre Avakas
\item Several softwares developped in Bordeaux: pari/GP, tulip, etc
\end{itemize}


\subsubsection*{Curriculum vitae}

% Curriculum of the personnel at this institution

\begin{participant}[type=leadPI,PM=12,salary=2500,gender=male]{Vincent Delecroix}
  \TOWRITE{VD}{Finalize CV}
  CNRS researcher at the Laboratoire Bordelais de recherche en informatique, Vincent
  Delecroix is a junior researcher in Number theory, Combinatorics and Dynamical systems.

  X publications.  He has several international collaboration (England, Mexico,
  United-States).

  \begin{itemize}
\item  Bobo
\item  cours Sage Bordeaux
\end{itemize}

Delecroix is a major developer of Sage in various components:
\begin{itemize}
\item  integers
\item  combinatorics
\item dynamical systems
\end{itemize}
\end{participant}
%%% Local Variables:
%%% mode: latex
%%% TeX-master: "../proposal"
%%% End:


%\input{CVs/Adrien.Boussicault.tex}
\begin{participant}[type=PI,PM=12,salary=7500,gender=male]{Karim Belabas}
  Professor of Mathematics in Bordeaux (France) since 2005, Karim Belabas is
a senior researcher in Number Theory, with particular interest in
computational and effective aspects. Karim has published about 25 articles
in international journals, including papers in Duke and Compositio (one of
which co-authored with Manjul Bhargava, 2014 Fields Medal), and
edited a book on ``Explicit methods in number theory''.

Karim was head of the Pure Math teaching department in Bordeaux from 2010 to
2014 and is vice-head of the Institut de Math\'ematiques de Bordeaux since
2015. He has held a grant from the French ANR worth $200$ kEuros (ALGOL
project, 2007--2011) and was part of a 2.5MEuros Marie-Curie Research
Training Network (GTEM project 2006--2010); he was responsible for three
deliverables and supervision of an early stage researcher during her PhD in
the Work Package ``Constructive Galois Theory''. He has (co-)organized 8
international conferences, including a special Trimester at IHP in 2004 and
an influential recurrent workshop on ``Explicit methods'' in Oberwolfach
(every two years since 2007) and 5 \PariGP Ateliers. He has (co-)supervised
11 PhD students and about 15 masters students.

Karim is a leading computational number theorist in France. He
is the project leader for the \PariGP free computer algebra system since
1995, which has had a major impact on hundreds of publications. He is one of
the system's main developers (about 60000 lines of code written, most of the
documentation, and 1300 bug-tracking tickets authored).
\end{participant}
%%% Local Variables:
%%% mode: latex
%%% TeX-master: "../proposal"
%%% End:

\begin{participant}[type=PI,PM=6,gender=male]{Bill Allombert}
  CNRS Ing\'enieur de Recherche. One of the main pari developers.
\end{participant}

%%% Local Variables:
%%% mode: latex
%%% TeX-master: "../proposal"
%%% End:

%?? others ??


\subsubsection*{Publications, products, achievements}

Some recent Publications :
\begin{enumerate}
\item 
Belabas, Karim; Friedman, Eduardo; Computing the residue of the Dedekind
zeta function.  Math. Comp. 84 (2015), no. 291, 357–369. 

\item
The PARI Group; PARI/GP version 2.7.0, Bordeaux, 2014,
http://pari.math.u-bordeaux.fr/.

\item
Belabas, Karim et al. Explicit methods in number theory. Rational points and
Diophantine equations, 179 pages, Panoramas et Synthèses 36, 179p., 2012.

\item
Allombert, Bill; Bilu, Yuri and Pizarro-Madariaga, Amalia;) CM-Points on
Straight Lines , to appear in "Analytic Number Theory" (dedicated do H. Maier),
Springer. 
\end{enumerate}



\subsubsection*{Previous projects or activities}

Current grants:
\begin{enumerate}
\item
 ANR PEACE (2012-2015)
    Goal: The discrete logarithm problem on algebraic curves is one of the rare
    contact points between deep theoretical questions in arithmetic geometry and
    every day applications. On the one side it involves a better understanding,
    from an effective point of view, of moduli space of curves, of abelian
    varieties, the maps that link these spaces and the objects they classify.
    On the other side, new and efficient algorithms to compute the discrete
    logarithm problem would have dramatic consequences on the security and
    efficiency of already deployed cryptographic devices. 

\item
ERC starting grant ANTICS (2011-2016) 
    Goal: "Rebuild algorithmic number theory on the firm grounds of theoretical
    computer science".
    Challenges: complexity (how fast can an algorithm be?), reliability
    (how correct should an algorithm be?), parallelisation.
\end{enumerate}

\subsubsection*{Significant infrastructure}
\TOWRITE{VD}{this still needs to be done}

Two center for computations: Plafrim and Avakas.
\end{sitedescription}


\begin{draft}
\vspace{1cm}\TOWRITE{VD}{Complete check list below -- delete completed items if you wish}

\begin{verbatim}
- [ ] checked that sum of person months put into finance request is
  the same as sum of person months associated with the Work Packages
  (in proposal.tex, as defined as part of the \begin{workpackage}"
  command.
  
  Take into account person months associated with work package 1, time
  of all staff to be hired and work on the project (including
  investigators). Figure 5 helps with a quick check of the sums over
  different work packages.

- [ ] completed site specific resource summary in resources.tex,
  including table of non-staff costs. This is compulsory (EU
  regulations) if the non-staff cost exceed 15% of the total cost, and
  is likely to be the case for most of the partners. We ask everybody
  to do it, to be consistent and show transparently how we have
  planned our total budget.

- [ ] Have all our tasks a designated lead institution? Check in the
  Work Packages that all the tasks you are involved in have a
  dedicated lead party. If the lead party is "USO", then use:
  \begin{task}[lead=USO]

- [ ] Have all our deliverables a designated lead institution [using
  the 'lead=' key]?

- [ ] In the "Members of the consortium section", have we addressed "a
  description of the legal entity and its main tasks, with an
  explanation of how its profile matches the tasks in the
  proposal"? See Entry for Paris Sud and Southampton as examples.

- [ ] In the Members of the consortium section, have we given
  descriptions of all the people we intend to hire (even if we don't
  know who that is yet). 
\end{verbatim}
\end{draft}

%KEY-MORE-TODOS


%%% Local Variables:
%%% mode: latex
%%% TeX-master: "../proposal"
%%% End:

%  LocalWords:  sitedescription th eorie des nombres Plafrim mesocentre Avakas developped
%  LocalWords:  subsubsection Belabas Synthèses Allombert Bilu Pizarro-Madariaga Maier
%  LocalWords:  parallelisation TOWRITE
